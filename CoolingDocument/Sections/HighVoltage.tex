
\section{Endcap High Voltage and Sensors in R3-R5}
\label{highvoltage}

The nominal thermoelectric model of an endcap module consists of one sensor connected in parallel with
an HVMUX and in series with a 10~k$\Omega$ filter (technically two 5k resistors).
Note, however, that endcap modules R3, R4 and R5 each contain two sensors, each of which has its own
HVMUX and series resistors. Because leakage current is proportional to the area of a sensor,
the effect of replacing a single sensor with a pair of sensors having the same total area is to split
the leakage current between the two sensors. Therefore, the total leakage current $I_S$ and sensor
power $Q_S$ are the same as in the single-sensor case.

However, these modules have two 10~k$\Omega$ serial resistors, each with half of the total
leakage current, and two HVMUX resistors, meaning the power dissipated is different from the nominal
case:

\begin{align}
P_{RHV}(I_S) ~=~& R_{HV}I_S^2~~\rightarrow~~ 2\times R_{HV}\left(\frac{I_S}{2}\right)^2 \\
P_{HVMUX} ~=~& \frac{ V^2_{bias} }{R_{HVMUX} + R_{HV}} ~~\rightarrow~~ 
             \frac{2V^2_{bias} }{R_{HVMUX} + R_{HV}}.
\end{align}

Thus, when treating the sensors in R3, R4, and R5, thermally the two sensors are treated as one sensor
(see Section~\ref{two_sensors_thermal_treat}, and electrically they are treated as one sensor with the
adjustments described above.
