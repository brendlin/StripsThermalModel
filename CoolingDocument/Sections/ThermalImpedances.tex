
\newcommand{\highlight}[1]{{\color{BrickRed}\textbf{#1}}}

\section{Collecting Power Inputs for the Endcap Petal}

Table~\ref{tab:power_numbers} details the current, voltage, and power specifications for each
component. Most of these numbers come from Graham and Georg's thermal model. Most are similar to
Sergio's numbers, with some differences highlighted in red.

\def\tid{\ensuremath{^\text{TID}}\xspace}
\def\eff{\ensuremath{\varepsilon}}
\def\pfeast{\ensuremath{\frac{(1-\eff)}{\eff}(P_\text{ABC}+P_\text{HCC})}}
%
\let\arraystretcha\arraystretch
\renewcommand\arraystretch{1.2} % 1.6
\begin{table}[h]
\begin{center}
\adjustbox{max width=\textwidth}{ %% just before tabular
\begin{tabular}{|l|c|c|c|c|c|c|} \hline
\multirow{2}{*}{Description} & input voltage & \multicolumn{3}{c|}{Specifications for 1 component} & $n$ components & Total power \\
              & [V]           & current [A]           & power [W]                   & eff   & per module (1 side) & (1 side) [W]    \\ \hline
AMAC 1.5V     & 1.5           & 0.045                 & 0.0675                      &       & --                  &                 \\
AMAC 3.0V     & 3.0           & 0.002                 & 0.006                       &       & --                  &                 \\
Total AMAC    & --            & --                    & 0.0735                      &       & 1                   & 0.0735          \\ \hline
ABC (digital) & 1.5           & 0.035                 & 0.0525                      &       & --                  &                 \\
ABC (analog)  & 1.5           & 0.066                 & 0.099                       &       & --                  &                 \\
Total ABC     & --            & 0.101                 & \highlight{0.1515}          &       & 21$^*$              & 3.1815$^*$\tid  \\ \hline
HCC (digital) & 1.5           & 0.125                 & 0.1875                      &       & --                  &                 \\
HCC (analog)  & 1.5           & 0.075                 & 0.1125                      &       & --                  &                 \\
Total HCC     & --            & 0.200                 & \highlight{0.3}             &       & 2$^*$               & 0.6$^*$\tid     \\ \hline
FEAST (ABC,HCC) & --          &                       & \pfeast                     & 75\%  & --                  & 1.2605$^*$\tid  \\
``FEAST'' AMAC regulators & --&                       & see below                   &       & --                  & 0.415           \\
Total FEAST   & --            &                       & see below                   &       & 1                   & 1.6755$^*$\tid  \\ \hline
Total Module (R1)  & --       &                       &                             &       &                     & 5.53$^*$        \\ \hline
\multicolumn{7}{|c|}{} \\[-2mm]
\multicolumn{7}{|c|}{EOS} \\ \hline
VTRx: lpGBTx  & 1.2           & 0.625                 & 0.750                       &       & --                  &                 \\
VTRx: GBLD 1.2V & 1.2         & 0.0095                & 0.0114                      &       & --                  &                 \\
VTRx: GBLD 2.5V & 2.5         & 0.018                 & 0.045                       &       & --                  &                 \\
Total VTRx    &               &                       & 0.8064                      &       & 1                   & 0.8064          \\         
GBTIA         & 2.5           & 0.053                 & 0.1325                      &       & 1                   & 0.1325          \\
FEAST         &               &                       &                             &       & 0.5$^\dagger$       & 0.35$^\dagger$  \\
DCDC2         &               &                       &                             & 88\%  & 0.5$^\dagger$       & 0.104$^\dagger$ \\ \hline
Total EOS     &               &                       &                             &       &                     & \highlight{1.4} \\
EOS both sides&               &                       &                             &       &                     & \highlight{2.8} \\
\hline \end{tabular}
} %% resizebox after tabular
\end{center}
\caption{Endcap module inputs. Starred ($^*$) values are representative and taken from Endcap R1. Values
with \tid next to them are affected by the TID bump in one way or another. The 75\% FEAST efficiency 
is representative only; in reality it is temperature- and current-dependent (also TID-dependent?).
}
\label{tab:power_numbers}
\end{table}
\let\arraystretch\arraystretcha

Some notes on the numbers in the table:
\begin{itemize}
\item HCC and ABC power numbers correspond to unirradiated values.
\item The ABC power numbers (from Georg/Graham) differ from Sergio's sheet by 1.7\%.
\item \highlight{The HCC power numbers (from Georg/Graham) differ from Sergio's sheet by 38\%.}
\item \tid The FEAST is affected by the TID bump insofar as its power is determined by the ABC and HCC.
  In the above it is assumed to be 75\%, but it is temperature and current-dependent.
\item The total power (before irradiation, before TID bump) of Module R1 represents
all components excluding HV and tape losses, which are small in comparison.
\item For the EOS, the total power for both sides is simply double the power of one side.
\item $^\dagger$ The FEAST and DCDC2 exist on one side only, and power the EOS cards on both petal
sides (hence ``0.5 per side''). If both EOSes are powered, then the power dissipatd by the FEAST and
DCDC2 is twice the power listed above.
\item For $n=1$ lpGBTx ASICs (corresponding to the barrel long strips and the endcaps),
and assuming a FEAST efficiency $\varepsilon_\text{FEAST}=0.75$,
the total power is 1.4~W per EOS side.
\item For $n=2$ lpGBTx ASICs (corresponding to the barrel short strips),
\highlight{the total power is 2.6~W per EOS side (differs from the TDR, which says 3~W)}.
\end{itemize}

%% \subsection{The End-of-Substructure (EOS)}
%% \begin{itemize}
%% \item DCDC2 converter: $P=0.208$~W (see calculation below)
%% \item FEAST: $P=1.12$~W (see calculation below)
%% \item VTRx
%%   \begin{itemize}
%%     \item GBTIA: $I=53$~mA; 2.5~V; $P=0.1325$~W
%%     \item GBLD$_{2.5V}$: $I=18$~mA 2.5~V; $P=0.045$~W
%%   \end{itemize}
%% \item lpGBTx: $I=625$~mA; 1.2~V; $P=0.75$~W; powered by DCDC2
%% \item GBLD$_{1.2V}$: $I=9.5$~mA; 1.2~V; $P=0.0114$~W; powered by DCDC2
%% \end{itemize}

The total power of the EOS (one side) is given by:
\begin{equation}
P_\text{EOS} = \frac{1}{\varepsilon_\text{FEAST}}\times
  \left( \frac{1}{\varepsilon_\text{DCDC2}} (n P_\text{lpGBTx} + n P_\text{GBLD1.2}) + n P_\text{GBLD2.5} + P_\text{GBTIA} \right)
\end{equation}
% (((.750*2+0.0114*2)/0.88)+0.1325+0.045*2)/0.75 = 2.60
% (((.750*1+0.0114*1)/0.88)+0.1325+0.045*1)/0.75 = 1.40

As can probably be inferred from above, the power in the EOS FEAST is:
\begin{equation}
P^\text{EOS}_\text{FEAST} = \frac{(1-\varepsilon_\text{FEAST})}{\varepsilon_\text{FEAST}}\times
  \left( \frac{1}{\varepsilon_\text{DCDC2}} (n P_\text{lpGBTx} + n P_\text{GBLD1.2}) + n P_\text{GBLD2.5} + P_\text{GBTIA} \right)
\end{equation}

The power in the EOS DCDC2 converter is:
\begin{equation}
P^\text{EOS}_\text{DCDC2} = \frac{(1-\varepsilon_\text{DCDC2})}{\varepsilon_\text{DCDC2}} \left(n P_\text{lpGBTx} + n P_\text{GBLD1.2}\right)
\end{equation}


The power dissipated by the AMAC regulators is given by the current in the AMAC components multiplied
by the voltage drop in the regulators:
\begin{equation}
P_\text{regulator} = I^{1.5V}_\text{AMAC}\left(  10.5V - 1.5V \right)
                   + I^{3.0V}_\text{AMAC}\left(  10.5V - 3.0V \right)
\label{eq:amac_regulator}
\end{equation}



\section{ Extracting thermal impedances using FEA simulations}

\subsection{Setup of the FEA Simulation to extract thermal impedances}

Representative power numbers are used to power each component. In each simulation run, all instances
of one type of component (e.g. the ABCs) are powered on in all six modules, keeping the rest off.
Average temperatures are measured for the following components (see below): HCC, ABC, FEAST, sensor.

\def\thcc{\ensuremath{\overline{T}_\text{nHCC}}}
\def\tabc{\ensuremath{\overline{T}_\text{nABC}}}
\def\tfeast{\ensuremath{\overline{T}_\text{FEAST}}}
\def\tsensor{\ensuremath{T_\text{sensor}}}
\def\Rm{\ensuremath{{\text{R}m}}}

\begin{itemize}
\item FEAST: a $3\times3$~mm $\times~350~\mu$m chip inside the shield box.
  \begin{itemize}
  \item The FEAST has an additional power term due to regulators for the AMAC -- see below.
  \end{itemize}
\item AMAC: a $3\times3$~mm $\times~350~\mu$m chip, roughly in the center of the power board
  \begin{itemize}
    \item The AMAC is powered using regulators located in the FEAST chip, with a power dissipation
      of the regulators corresponding to the voltage drop in the regulator (0.415~W) -- see Eq.~\ref{eq:amac_regulator}.
  \end{itemize}
\item HVMUX: Ignore for now
\item EOS (\highlight{3.20~W total, including both sides -- see Table~\ref{tab:power_numbers}.}) % (\highlight{3.03~W total, for both sides}):
  \begin{itemize}
  \item \highlight{Placement of these EOS power sources: ???}
%%     \item 1 lpGBT per side (\highlight{0.750~W $\times$ 2 sides})
%%     \item ``Rest'' (1 VTRX optical link per side?) (\highlight{.185~W $\times$ 2 sides}) ???
%%     \item FEAST on a single side (\highlight{1.12~W})
%%     \item DCDC2 converter on a single side (\highlight{0.21~W})
  \end{itemize}
\item Cooling: Constant 8000~W/m$^{2}$K, $-30$~C, no convection
\end{itemize}




\subsection{FEA Simulation Runs to extract thermal impedances}

For the extraction of the thermal impedances, a simplified set of input power parameters are used,
summarized in Table~\ref{tab:simulation_runs}.
In the FEA, the power is distributed over all 6 surfaces (\highlight{different from barrel treatment}).

\let\arraystretcha\arraystretch
\renewcommand\arraystretch{1.4} % 1.6
\begin{table}[h!]
\footnotesize
\begin{center}
\adjustbox{max width=\textwidth}{ %% just before tabular
\begin{tabular}{|l|l|l|l|} \hline
Simulation \# & Description                        & Input parameters           \\ \hline
1             & All HCCs powered on, rest off      & $P_\text{HCC}=0.413$~W     \\
2             & All ABCs powered on, rest off      & $P_\text{ABC}=0.149$~W     \\
3             & All FEASTs powered on, rest off    & $P_\text{FEAST}=1.5$~W$^*$ \\ \hline
\multicolumn{3}{|c|}{Extended simulations} \\ \hline
4             & Tape ``powered'' on, rest off      & skip for now \\
5             & HVMUX powered on, rest off         & skip for now \\
6             & R$_\text{HV}$ powered on, rest off & skip for now \\
7             & EOS                                & skip for now \\ % $P_\text{EOS}=3.03$~W$^{**}$
\hline \end{tabular}
} %% resizebox after tabular
\end{center}
\caption{ Description of the 7 thermal simulations required to obtain the thermal impedances.
}
\label{tab:simulation_runs}
\end{table}
\let\arraystretch\arraystretcha

$^*$ The actual nominal power of the FEAST varies for each module; however, for the simulation to extract
the thermal impedances, the power is set to 1.5~W for all FEASTs in the petal.\\






\subsection{Measurements performed in each run}

%% For each component, the temperature measured is the average of the top surface nodes of all
%% components of a given type, in a given module.

The average, min and max temperatures are taken over the volume of the elements (\highlight{different from barrel treatment}).
There are 45 measurements per simulation run in total.

The average temperatures can be measured either \highlight{on one side of the petal, or as the average of components on both sides 
of the petal}.

\begin{itemize}
\item $(\thcc)_\Rm$: The average HCC temperature of the $n$ HCCs in the module, for each module R$m$ (R0, R1, ... R5) (6~total)
\item $(\tabc)_\Rm$: The average ABC temperature of the $n$ ABCs in the module, for each module R$m$ (6~total)
\item $(\tfeast)_\Rm$: The temperature of the FEAST in the module, for each module R$n$ (6~total)
\item \tsensor, taken for R0, R1, R2, R3\_left, R3\_right, R4\_left, R4\_right, R5\_left, R5\_right (9~total) (27~total measurements):
\begin{itemize}
  \item $\tsensor^\text{Avg}$: The average sensor temperature taken over the volume of the sensor
  \item $\tsensor^\text{Max}$: The maximum sensor temperature in the module, for each module
  \item $\tsensor^\text{Min}$: The minimum sensor temperature in the module, for each module
\end{itemize}
\end{itemize}




\section{The Linear Model}

\subsection{Additional parameters in the linear model}

The number of ABCs, HCC, and the sensor area are below,
as are the FEAST currents before irradiation (pre-TID bump).
%
\let\arraystretcha\arraystretch
\renewcommand\arraystretch{1.1} % 1.6
\begin{table}[h]
\footnotesize
\begin{center}
\adjustbox{max width=\textwidth}{ %% just before tabular
\begin{tabular}{|l|r|r|r|r|r|} \hline
Module & nABC & nHCC & Sensor area (cm$^2$) & $I_\text{FEAST}$ [A] & $P_\text{FEAST}$ \\ \hline
R0     &   17 &    2 &                 92.0 &              2.12 & 1.0585 \\
R1     &   21 &    2 &                 91.0 &              2.52 & 1.2605 \\
R2     &   12 &    2 &                 76.0 &              1.61 & 0.806  \\
R3     &   28 &    4 &                164.0 &              3.63 & 1.814  \\
R4     &   16 &    2 &                178.0 &              2.02 & 1.008  \\
R5     &   18 &    2 &                186.0 &              2.22 & 1.109  \\
\hline \end{tabular}
} %% resizebox after tabular
\end{center}
\caption{Endcap module inputs. The FEAST current is calculated from the HCC and ABC values in
Table~\ref{tab:power_numbers}, and given nABC and nHCC in each module. The FEAST power is calculated
assuming a 75\% efficiency, and does not include the power due to the AMAC regulators.}
\label{tab:spurious_signal_main}
\end{table}
\let\arraystretch\arraystretcha


\subsection{Other assumed quantities}

Assumed quantities are below:
\begin{itemize}
\item $R_\text{EOS}=15.0$~K/W (guessed by Georg/Graham)
\item $R_\text{sensor}=0.02$~K/W (guessed by Georg/Graham)
\item $R_\text{tape}=0.01$~K/W per module (i.e. there are 6 such resistors in the endcap) (worst-case number)
%% \item EOS DCDC2 efficiency: 0.88
%% \item $V_\text{hybrid}=1.5$~V
%% \item $I^\text{digital}_\text{HCC}=0.125$~A (before TID damage; TID-dependent)
%% \item $I^\text{analog}_\text{HCC}=0.075$~A
%% \item $I^\text{digital}_\text{ABC}=0.035$~A (before TID damage; TID-dependent)
%% \item $I^\text{analog}_\text{ABC}=0.066$~A
\end{itemize}

%% The AMAC:
%% \begin{itemize}
%% \item $V^{1.5V}_\text{AMAC} = 1.5$~V
%% \item $I^{1.5V}_\text{AMAC} = 0.045$~A
%% \item (Efficiency $\varepsilon^{1.5V}_\text{AMAC} = 0.65$\% -- not used in model)
%% \item $V^{3.0V}_\text{AMAC} = 3.0$~V
%% \item $I^{3.0V}_\text{AMAC}= 0.002$~A
%% \item (Efficiency $\varepsilon^{3.0V}_\text{AMAC} = 0.65$\% -- not used in model)
%% \end{itemize}

