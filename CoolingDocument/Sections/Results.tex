
\newcommand{\mry}[2]{
\multirow{#1}{*}{Year #2}
}

\section{Results}

\subsection{Description of Output}
\subsubsection{Module-level quantities}
%% \let\itemsepa\itemsep
%% \renewcommand\itemsep{1.0}
Avearage temperatures are reported on each module for each of the following: sensor, ABC, HCC, Feast,
and EOS.

Reported power values for each module: 
\begin{itemize}
\setlength\itemsep{0.0em}
\item Sensor Q (one side)
\item EOS power (one side)
\item LV tape power loss due to items on module (one side)
\item Cumulative LV tape power loss (one module one side):
Power loss in the tape associated to this module, due to the current from items on the module, plus
current from previous modules.
\item HV power serial resistors (one side)
\item HV power parallel resistor (one side)
\item Total HV power (including leakage) (one side)
\item Module power (one side) (no EOS): front-end + HV + tape
\end{itemize}

Reported current values for each module:
\begin{itemize}
\setlength\itemsep{0.0em}
\item Sensor (leakage) current (one module side)
\item LV tape current (one side) due to items on module
\item LV tape current (one side)
\item Tape current load due to EOS (one side)
\item ABC and HCC digital current
\item Hybrid 0 (1,2,3) current (if applicable)
\item FEAST current (load per Feasts, in case there is more than one Feast)
\item FEAST current (input, both Feasts, in case there is more than one Feast.)
\end{itemize}

Other reported module-level values:
\begin{itemize}
\setlength\itemsep{0.0em}
\item Feast efficiency
\item Sensor Q headroom factor
\item Coolant temperature headroom
\end{itemize}
%% \let\itemsep\itemsepa

\subsubsection{Petal-level quantities}

The following items are reported for petals at each disk position:
\begin{itemize}
\setlength\itemsep{0.0em}
\item Total power in petal located at Disk $n$ (including the EOS) (one petal side)
\item Total sensor Q in petal located at Disk $n$ (one side)
\item HV power in petal located at Disk $n$ (one side)
\item LV tape current in petal located at Disk $n$ (including the EOS) (one side)
\item Total sensor (leakage) current in petal located at Disk $n$ (one side)
\end{itemize}

\subsubsection{Full system quantities}

\begin{itemize}
\setlength\itemsep{0.0em}
\item Total power in both endcaps, including cable losses -- modeled as 
5\%$\times$5\% for cable losses from the service modules and cable losses from PP1 to USA15 (10.25\% total loss).
More realistic cable losses are under construction.
\item Total HV power (sensor + resistors) in both endcaps
\end{itemize}

\subsection{Nominal and Pessimistic scenarios}

Table~\ref{results_summary} shows the power estimates, properties (temperature) of the sensor and
FEAST, as well as HV and LV currents. Two scenarios are shown: one in which no safety factors are
applied, and one in which all safety factors (Fluence, thermal impedance, digital and analog current,
TID parameterization, and HV bias) are applied. In both cases, a wide range of cooling scenarios
(from $-35^\circ$C to $-35^\circ$C coolant temperature) are shown, as well as the ``ramp'' scenario
in which the temperature is ramped down from $0^\circ$C to $-35^\circ$C over the course of 9 years
(the same as used in the barrel).

In the nominal scenario, the Endcap System has a total power (including cable losses) between 32.1
and 38.3 kW (it varies due to the effect of the TID bump in the electronics).
In the worst-case safety factor scenario, the total power is 36.8-46.7 kW.

In places where thermal runaway occurs, the R3 module reaches thermal runaway before the others
(although at higher temperatures in the pessimistic safety-factor scenario, all endcap modules
eventually run away).

It is also important to note that runaway always occurs near the end of the detector life, rather than
at the TID bump.

\let\arraystretcha\arraystretch
\renewcommand\arraystretch{1.15} % 1.6

\begin{table}[ht]
\begin{subtable}[t]{.99\linewidth}
\begin{centering}\adjustbox{max width=\textwidth}{ %% just before tabular
\begin{tabular}{|l|l|r|r|r|r|r|r|} \hline % data_below
\multirow{5}{*}{Safety Factors} & Fluence                                        &           0.0 &           0.0 &           0.0 &           0.0 &           0.0 &           0.0 \\ 
                                & $R_{T}$                                        &           0.0 &           0.0 &           0.0 &           0.0 &           0.0 &           0.0 \\ 
                                & $I_D$                                          &           0.0 &           0.0 &           0.0 &           0.0 &           0.0 &           0.0 \\ 
                                & $I_A$                                          &           0.0 &           0.0 &           0.0 &           0.0 &           0.0 &           0.0 \\ 
                                & TID parameterization                           &       nominal &       nominal &       nominal &       nominal &       nominal &       nominal \\ \hline
\multirow{2}{*}{HV, Cooling}    & Voltage [V]                                    &         500.0 &         500.0 &         500.0 &         500.0 &         500.0 &         500.0 \\ 
                                & Cooling [$^\circ$C]                            &    flat $-$35 &    flat $-$30 &    flat $-$25 &    flat $-$20 &    flat $-$15 &    ramp $-$35 \\ \hline
\multirow{3}{*}{Endcap System}  & Minimum power [kW]                             &          32.1 &          32.2 &          32.3 &          32.4 &   \bf Runaway &          32.9 \\ 
                                & Maximum power [kW]                             &          38.3 &          37.3 &          36.6 &          40.3 &   \bf Year 11 &          36.2 \\ 
                                & Maximum $P_\text{HV}$ [kW]                     &          1.08 &          1.96 &          3.65 &          7.22 &      \bf (R3) &          2.10 \\ \hline
\multirow{2}{*}{Petal-level}    & Max petal LV $I_\text{tape}$ load [A]          &          4.32 &          4.20 &          4.11 &          4.03 &   \mry{2}{11} &          3.93 \\ 
                                & Max petal power [W]                            &          45.8 &          44.6 &          44.7 &          52.5 &               &          43.7 \\ \hline
R0                              & \multirow{6}{*}{Min/Max Power [W]}             &   5.31 / 8.01 &   5.33 / 7.56 &   5.35 / 7.20 &   5.37 / 7.26 &   \mry{6}{11} &   5.39 / 6.43 \\ 
R1                              &                                                &   6.36 / 8.75 &   6.38 / 8.36 &   6.41 / 8.05 &   6.44 / 8.56 &               &   6.46 / 7.53 \\ 
R2                              &                                                &   4.19 / 5.23 &   4.20 / 5.07 &   4.21 / 4.95 &   4.23 / 5.64 &               &   4.27 / 4.86 \\ 
R3                              &                                                &   9.36 / 11.6 &   9.39 / 11.3 &   9.42 / 11.4 &   9.45 / 14.4 &               &   9.57 / 10.9 \\ 
R4                              &                                                &   5.20 / 6.45 &   5.22 / 6.27 &   5.23 / 6.32 &   5.25 / 7.49 &               &   5.33 / 6.21 \\ 
R5                              &                                                &   5.72 / 7.01 &   5.74 / 6.82 &   5.76 / 6.68 &   5.78 / 7.38 &               &   5.87 / 6.71 \\ \hline
\multirow{5}{*}{Components}     & Max $I_{HV}$ per module [mA]                   &    0.965 (R3) &     1.85 (R3) &     3.65 (R3) &     8.40 (R3) &   \mry{5}{11} &     1.95 (R3) \\ 
                                & Max sensor T [$^\circ$C], Y1                   &      -27 (R3) &    -21.9 (R3) &    -16.9 (R3) &    -11.9 (R3) &               &     8.21 (R3) \\ 
                                & Max sensor T [$^\circ$C], Y14                  &    -26.5 (R3) &    -21.1 (R3) &    -15.2 (R3) &    -7.56 (R3) &               &    -26.5 (R3) \\ 
                                & Max sensor T [$^\circ$C], Max                  &      -25 (R3) &    -20.3 (R3) &    -15.2 (R3) &    -7.56 (R3) &               &     8.50 (R3) \\ 
                                & Max $T_\text{Feast}$                           &     28.5 (R1) &     30.3 (R1) &     32.9 (R1) &     36.2 (R1) &               &     49.3 (R1) \\ \hline
\multirow{2}{*}{Headroom}       & Min $Q_{sensor}$ Headroom [$Q_{S,crit}/Q_{S}$] &     7.64 (R3) &     4.27 (R3) &     2.37 (R3) &     1.23 (R3) &   \mry{2}{11} &     4.37 (R3) \\ 
                                & Min Coolant Temp Headroom [$^\circ$C]          &     19.0 (R3) &     13.8 (R3) &     8.35 (R3) &     2.01 (R3) &               &     15.3 (R3) \\ 
\hline\end{tabular}
} %% resize box after tabular
\end{centering}
\caption{Summary of nominal (no safety factors) scenarios, with different coolant temperatures.}
\end{subtable}

\vspace{5mm}

\begin{subtable}[t]{.99\linewidth}
\begin{centering}\adjustbox{max width=\textwidth}{ %% just before tabular
\begin{tabular}{|l|l|r|r|r|r|r|r|} \hline % data_below
\multirow{5}{*}{Safety Factors} & Fluence                                        &           0.5 &          0.5 &           0.5 &           0.5 &           0.5 &           0.5 \\ 
                                & $R_{T}$                                        &           0.2 &          0.2 &           0.2 &           0.2 &           0.2 &           0.2 \\ 
                                & $I_D$                                          &           0.2 &          0.2 &           0.2 &           0.2 &           0.2 &           0.2 \\ 
                                & $I_A$                                          &          0.05 &         0.05 &          0.05 &          0.05 &          0.05 &          0.05 \\ 
                                & TID parameterization                           &   pessimistic &  pessimistic &   pessimistic &   pessimistic &   pessimistic &   pessimistic \\ \hline
\multirow{2}{*}{HV, Cooling}    & Voltage [V]                                    &         700.0 &        700.0 &         700.0 &         700.0 &         700.0 &         700.0 \\ 
                                & Cooling [$^\circ$C]                            &    flat $-$35 &   flat $-$30 &    flat $-$25 &    flat $-$20 &    flat $-$15 &    ramp $-$35 \\ \hline
\multirow{3}{*}{Endcap System}  & Minimum power [kW]                             &          36.8 &         36.9 &   \bf Runaway &   \bf Runaway &   \bf Runaway &          37.8 \\ 
                                & Maximum power [kW]                             &          46.7 &         45.3 &   \bf Year 10 &   \bf Year  7 &   \bf Year  6 &          45.3 \\ 
                                & Maximum $P_\text{HV}$ [kW]                     &          3.10 &         6.28 &   \bf (R3)    &   \bf (all)   &   \bf (all)   &          6.52 \\ \hline
\multirow{2}{*}{Petal-level}    & Max petal LV $I_\text{tape}$ load [A]          &          5.34 &         5.15 &   \mry{2}{10} &    \mry{2}{7} &    \mry{2}{6} &          4.57 \\ 
                                & Max petal power [W]                            &          56.8 &         55.9 &               &               &               &          56.7 \\ \hline
R0                              & \multirow{6}{*}{Min/Max Power [W]}             &   6.07 / 10.6 &  6.09 / 9.95 &   \mry{6}{10} &    \mry{6}{7} &    \mry{6}{6} &   6.24 / 8.01 \\ 
R1                              &                                                &   7.31 / 11.5 &  7.35 / 11.0 &               &               &               &   7.55 / 9.44 \\ 
R2                              &                                                &   4.78 / 6.55 &  4.79 / 6.28 &               &               &               &   4.88 / 6.10 \\ 
R3                              &                                                &   10.7 / 14.1 &  10.7 / 15.8 &               &               &               &   11.0 / 15.7 \\ 
R4                              &                                                &   5.96 / 8.05 &  5.98 / 7.75 &               &               &               &   6.12 / 8.08 \\ 
R5                              &                                                &   6.56 / 8.99 &  6.58 / 8.61 &               &               &               &   6.76 / 8.40 \\ \hline
\multirow{5}{*}{Components}     & Max $I_{HV}$ per module [mA]                   &     2.48 (R3) &    6.51 (R3) &   \mry{5}{10} &    \mry{5}{7} &    \mry{5}{6} &     6.29 (R3) \\ 
                                & Max sensor T [$^\circ$C], Y1                   &      -24 (R3) &   -18.9 (R3) &               &               &               &     11.3 (R3) \\ 
                                & Max sensor T [$^\circ$C], Y14                  &    -22.1 (R3) &   -13.6 (R3) &               &               &               &    -22.1 (R3) \\ 
                                & Max sensor T [$^\circ$C], Max                  &    -20.4 (R3) &   -13.6 (R3) &               &               &               &     12.5 (R3) \\ 
                                & Max $T_\text{Feast}$                           &     75.2 (R1) &    75.3 (R1) &               &               &               &     79.9 (R1) \\ \hline
\multirow{2}{*}{Headroom}       & Min $Q_{sensor}$ Headroom [$Q_{S,crit}/Q_{S}$] &     2.09 (R3) &    1.11 (R3) &   \mry{2}{10} &    \mry{2}{7} &    \mry{2}{6} &     1.16 (R3) \\ 
                                & Min Coolant Temp Headroom [$^\circ$C]          &     6.73 (R3) &   0.943 (R3) &               &               &               &     1.53 (R3) \\ 
\hline\end{tabular}
} %% resize box after tabular
\end{centering}
\caption{Summary of worst-case safety factor scenarios, with different coolant temperatures.}
\end{subtable}
\caption{Comparing nominal and worst-case and safety factor scenarios.}
\label{results_summary}
\end{table}
\let\arraystretch\arraystretcha

\clearpage

\subsection{Interpretation and Main Takeaways}

Two particularly important quantities to monitor are the $Q_{sensor}$ headroom and the coolant
temperature headroom. The $Q_{sensor}$ headroom is the multiplicative factor
$[Q_{s,critical}/Q_{s,current}]$ by which the sensor $Q$ can increase before reaching thermal runaway
(when this value falls to 1, thermal runaway has occurred). The coolant temperature headroom is the
number of degrees C that you could raise the coolant temperature before thermal runaway occurs (the
headroom falls to 0 at thermal runaway). Even without any safety factors, the $Q_{sensor}$ headroom
is below 10; with all of the safety factors applied, the headroom is around 2. The temperature
headroom tells a similar story.

Two questions that can be asked is how to improve the sensor $Q$ headroom, and, since there are many
moving parts in the model, which methods of improving the sensor $Q$ headroom are most effective? The
answer can be addressed by evaluating the effect of each individual safety factor on the sensor $Q$
headroom (and/or the coolant headroom).

Table~\ref{detailed_safety_table} offers a detailed look at the effect of each individual safety
factor; the ``safety factor scenarios'' are ordered according to which scenario has the larger impact
on the minimum sensor $Q$ headroom (minimum sensor $Q$ headroom means the minimum among all of the sensors in
the endcap, at any point in time). According to the results in the table, the thermal impedance
$R_T$ (with a +20\% safety factor) has the largest impact on the sensor $Q$ headroom,
followed by the HV bias (the safety factor is 700 V), and finally the
currents of the ABC, HCC and AMAC\footnote{
The current is typically dominated by the ABC ($\sim$80\%), followed by the HCC ($\sim$20\%). The AMAC
contributes about 1\% to the total power.}.
Note that the fluence safety factor is out of our control, and the nature of the TID bump does not
have any impact on the sensor $Q$ headroom, since its effect is minimal at the detector end-of-life.

The outlook can be ``improved'' in two ways: first, by improving the accuracy of the safety factors.
In other words, if we can e.g. improve the endcap thermal model and build confidence that the thermal
impedance safety factor should be 10\% rather than 20\%, then our worst-case scenarios will have
larger sensor $Q$ headroom values. Another example would be to guarantee that the HV will never run
above 650V. Of course, the margin for error would not have changed---only our confidence that we can
run within that margin of error.

The outlook can also be improved by physical changes to the petal. This includes pipe rerouting, 
using more thermally conductive material, etc. to reduce the thermal impedance of the material
underneath the sensor. It could include making changes to the ABC chip to reduce the current.
These changes would have the effect of reducing the temperature of the sensors, thus increasing the
sensor $Q$ headroom.

\begin{table}[ht]
\begin{centering}\adjustbox{max width=\textwidth,max totalheight=\textheight}{ %% just before tabular
\begin{tabular}{|ccccc|cc|rrrr|r|r|} \hline % data_below
\multicolumn{5}{|c|}{Safety factor} & $V_{bias}$ & Cooling & Endcaps max & Endcaps max & Min sensor $Q$ & Min Coolant\\
Fluence & $R_{T}$ & $I_D$ & $I_A$ &   TID &   [V] & [$^\circ$C] & \multicolumn{1}{l}{power [kW]} & \multicolumn{1}{l}{HV [kW]} & \multicolumn{1}{l}{headroom} & \multicolumn{1}{l|}{headroom [$^\circ$C]} \\ \hline
0.0     &     0.0 &   0.0 &   0.0 & False & 500.0 &  $-$35 flat &      38.3 &    1.08 &      7.64 &       19.0 \\ 
0.5     &     0.0 &   0.0 &   0.0 & False & 500.0 &  $-$35 flat &      39.2 &    1.59 &      5.05 &       14.9 \\ 
0.5     &     0.0 &   0.0 &   0.0 &  True & 500.0 &  $-$35 flat &      41.3 &    1.59 &      5.05 &       14.9 \\ 
0.5     &     0.0 &   0.2 &  0.05 & False & 500.0 &  $-$35 flat &      44.9 &    1.75 &      4.45 &       13.8 \\ 
0.5     &     0.0 &   0.0 &   0.0 & False & 700.0 &  $-$35 flat &      39.4 &    2.32 &      3.60 &       11.6 \\ 
0.5     &     0.2 &   0.0 &   0.0 & False & 500.0 &  $-$35 flat &      38.9 &    1.86 &      3.44 &       11.4 \\ 
0.5     &     0.2 &   0.2 &  0.05 &  True & 700.0 &  $-$35 flat &      46.7 &    3.10 &      2.09 &       6.73 \\ \hline
0.0     &     0.0 &   0.0 &   0.0 & False & 500.0 &  $-$30 flat &      37.3 &    1.96 &      4.27 &       13.8 \\ 
0.5     &     0.0 &   0.0 &   0.0 & False & 500.0 &  $-$30 flat &      38.1 &    2.98 &      2.77 &       9.55 \\ 
0.5     &     0.0 &   0.0 &   0.0 &  True & 500.0 &  $-$30 flat &      39.8 &    2.98 &      2.77 &       9.55 \\ 
0.5     &     0.0 &   0.2 &  0.05 & False & 500.0 &  $-$30 flat &      43.6 &    3.31 &      2.43 &       8.34 \\ 
0.5     &     0.0 &   0.0 &   0.0 & False & 700.0 &  $-$30 flat &      38.3 &    4.37 &      1.97 &       6.29 \\ 
0.5     &     0.2 &   0.0 &   0.0 & False & 500.0 &  $-$30 flat &      37.9 &    3.58 &      1.85 &       5.76 \\ 
0.5     &     0.2 &   0.2 &  0.05 &  True & 700.0 &  $-$30 flat &      45.3 &    6.28 &      1.11 &      0.943 \\ \hline
0.0     &     0.0 &   0.0 &   0.0 & False & 500.0 &  $-$25 flat &      36.6 &    3.65 &      2.37 &       8.35 \\ 
0.5     &     0.0 &   0.0 &   0.0 & False & 500.0 &  $-$25 flat &      38.7 &    5.87 &      1.45 &       3.53 \\ 
0.5     &     0.0 &   0.0 &   0.0 &  True & 500.0 &  $-$25 flat &      38.7 &    5.87 &      1.45 &       3.53 \\ 
0.5     &     0.0 &   0.2 &  0.05 & False & 500.0 &  $-$25 flat &      43.9 &    6.64 &      1.24 &       2.05 \\ 
0.5     &     0.0 &   0.0 &   0.0 & False & 700.0 &  $-$25 flat &      42.2 &    9.07 &      1.03 &      0.297 \\ 
0.5     &     0.2 &   0.0 &   0.0 & False & 500.0 &  $-$25 flat & \multicolumn{4}{c|}{\bf Runaway Year 13} \\
0.5     &     0.2 &   0.2 &  0.05 &  True & 700.0 &  $-$25 flat & \multicolumn{4}{c|}{\bf Runaway Year 10} \\
\hline\end{tabular}
} %% resize box after tabular
\caption{Summary of all safety factor scenarios and their effect on the maximum power in the
endcaps, the maximum HV power, the minimum sensor headroom $[Q_{s,crit}/Q_s]$, and the coolant temperature headroom.}
\label{detailed_safety_table}
\end{centering}
\end{table}
