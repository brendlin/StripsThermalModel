
\newcommand{\mry}[2]{
\multirow{#1}{*}{Year #2}
}

\section{Results}

\subsection{Description of Output}
\label{description_of_output}
\subsubsection{Module-level quantities}
%% \let\itemsepa\itemsep
%% \renewcommand\itemsep{1.0}
Avearage temperatures are reported on each module for each of the following: sensor, ABC, HCC, BPOL12V,
and EOS.

Reported power values for each module: 
\begin{itemize}
\setlength\itemsep{0.0em}
\item Sensor Q (one side). {\bf If there are two sensors on a module, then this quantity is the sum of
their Q.}
\item EOS power (one side)
\item LV tape power loss due to items on module (one side)
\item Cumulative LV tape power loss (one module one side):
Power loss in the tape associated to this module, due to the current from items on the module, plus
current from previous modules. It does not include the power dissipated by tape in other modules.
\item HV power serial resistors (one side)
\item HV power parallel resistor (one side)
\item Total HV power (including leakage and resistors) (one side)
\item Module power (one side) (no EOS): front-end + HV + tape
\end{itemize}

Reported current values for each module:
\begin{itemize}
\setlength\itemsep{0.0em}
\item Sensor (leakage) current (one module side). {\bf If there are two sensors on a module, then this
quantity is the sum of their leakage currents.}
\item LV tape current (one side) due to items on module
\item LV tape current (one side)
\item Tape current load due to EOS (one side)
\item ABC and HCC digital current
\item Hybrid 0 (1,2,3) current (if applicable)
\item BPOL12V current (load per BPOL12V, in case there is more than one BPOL12V)
\item BPOL12V current (input, both BPOL12Vs, in case there is more than one BPOL12V.)
\end{itemize}

Other reported module-level values:
\begin{itemize}
\setlength\itemsep{0.0em}
\item BPOL12V efficiency
\item Sensor Q headroom factor
\item Coolant temperature headroom
\end{itemize}
%% \let\itemsep\itemsepa

\subsubsection{Petal-level quantities}

The following items are reported for petals at each disk position:
\begin{itemize}
\setlength\itemsep{0.0em}
\item Total power in petal located at Disk $n$ (including the EOS) (one petal side)
\item Total sensor Q in petal located at Disk $n$ (one side)
\item HV power in petal located at Disk $n$ (one side)
\item ``LV tape power load (incl. tape losses)'': Total power dissipated by the petal tape and LV
FE components (e.g. total module LV power)
\item $\Delta V_\text{tape}$: voltage drop through the petal tape alone.
\item $I_\text{tape}$: LV tape current in petal located at Disk $n$ (including the EOS) (one side)
\item Total sensor (leakage) current in petal located at Disk $n$ (one side)
\end{itemize}

\subsubsection{Full system quantities}

\begin{itemize}
\setlength\itemsep{0.0em}
%% \item Total power in both endcaps, including cable losses -- modeled as
%% 5\%$\times$5\% for cable losses from the service modules and cable losses from PP1 to USA15 (10.25\% total loss).
%% More realistic cable losses are under construction.
\item Power in both endcaps (LV+HV), excluding services (e.g. excluding cables, patch-panels)
\item Cooling system power (e.g. with Type I cables and PP1, though no power from PP1 is modeled)
\item Wall power (e.g. including all services and power supplies). This means the ``Cooling system power''
  above, plus LV and HV type-II/III/IV cable power losses, and LV and HV losses in PP2.
\item Total HV power (sensor + resistors) in both endcaps
\item Service power (e.g. power lost in Type I, II, III and IV cables and PP2) % add HV tape later
\end{itemize}

\subsubsection{New cabling and tape quantities}

\begin{itemize}
\setlength\itemsep{0.0em}
\item $\Delta V_{HV}$ (filters, \sout{tape}, cables, PP2): HV voltage drop from type
III/IV cables, PP2, type I/II cables, and on-module HV filter. (e.g. from PP4 out to sensor in).
This is a module-level quantity since $\Delta V$ of the HV filter differs per module.
\item LV round-trip $\Delta V$ from PP2 (meaning type I/II cables only)
\item LV $V_\text{out}$ at PP2. This includes voltage drops from the type I and II cables, in the
  petal tape, and the module components (BPOL12V/LinPOL12V).
\end{itemize}

\subsection{Nominal and Pessimistic scenarios}

Table~\ref{results_summary} shows the power estimates, properties (temperature) of the sensor and
BPOL12V, as well as HV and LV currents. Two scenarios are shown: one in which no safety factors are
applied, and one in which all safety factors (Fluence, thermal impedance, digital and analog current,
TID parameterization, and HV bias) are applied. In both cases, a wide range of cooling scenarios
(from $-35^\circ$C to $-15^\circ$C coolant temperature) are shown, as well as the ``ramp'' scenario
in which the temperature is ramped down from $0^\circ$C to $-35^\circ$C over the course of 9 years
(the same as used in the barrel).

In the nominal scenario (no safety factors), the Endcap System has a Wall power (including cable losses) between 39.0
and 46.4 kW (it varies due to the effect of the TID bump in the electronics).
In the worst-case safety factor scenario, the wall power is 45.5-58.3 kW.

In places where thermal runaway occurs, the R3 module reaches thermal runaway before the others
(although at higher temperatures in the pessimistic safety-factor scenario, all endcap modules
eventually run away).

It is also important to note that runaway always occurs near the end of the detector life, rather than
at the TID bump.

\clearpage
\newgeometry{top=20mm, bottom=10mm, left=0.75in, right=0.75in,
            foot=0mm, marginparsep=0mm}

\let\arraystretcha\arraystretch
\renewcommand\arraystretch{1.15} % 1.6

\begin{table}[ht]
\begin{subtable}[t]{.99\linewidth}
\begin{centering}\adjustbox{max width=\textwidth}{ %% just before tabular
\begin{tabular}{|l|l|r|r|r|r|r|r|} \hline % data_below
\multirow{5}{*}{Safety Factors} & Fluence                                                               & \multicolumn{6}{c|}{    0.0} \\
                                & $R_{T}$                                                               & \multicolumn{6}{c|}{    0.0} \\
                                & $I_D$                                                                 & \multicolumn{6}{c|}{    0.0} \\
                                & $I_A$                                                                 & \multicolumn{6}{c|}{    0.0} \\
                                & TID parameterization                                                  & \multicolumn{6}{c|}{nominal} \\ \hline
\multirow{2}{*}{HV, Cooling}    & Voltage [V]                                                           &         500.0 &         500.0 &         500.0 &         500.0 &         500.0 &         500.0 \\
                                & Cooling [$^\circ$C]                                                   &    flat $-$35 &    flat $-$30 &    flat $-$25 &    flat $-$20 &    flat $-$15 &    ramp $-$35 \\ \hline
\multirow{3}{*}{Endcap System}  & Total LV+HV, no services                                              &     28.1/32.6 &     28.2/31.8 &     28.3/31.7 &     28.4/35.0 &               &     28.7/31.2 \\
\multirow{3}{*}{Min/Max}        &  + type 1 cables, PP1 (Cooling system power)                          &     28.9/33.7 &     29.0/32.9 &     29.1/32.5 &     29.2/35.9 &   \bf Runaway &     29.6/32.2 \\
\multirow{3}{*}{Power [kW]}     &  + all services and power supplies (Wall power)                       &     39.0/46.4 &     39.2/45.0 &     39.3/44.0 &     39.5/46.2 &   \bf Year 12 &     39.8/43.0 \\
                                & Service power only                                                    &     10.9/13.8 &     11.0/13.2 &     11.1/12.8 &     11.1/12.5 &      \bf (R3) &     11.0/12.1 \\
                                & Maximum $P_\text{HV}$ [kW]                                            &          1.11 &          1.95 &          3.56 &          6.81 &               &          2.08 \\ \hline
Petal-level                     & Max petal power (LV+HV) [W]                                           &          42.9 &          41.8 &          43.0 &          49.8 &   \mry{1}{12} &          41.4 \\ \hline
\multirow{3}{*}{Petal LV tape}  & Max LV tape power load (incl. tape losses) [W]                        &          40.8 &          39.6 &          38.7 &          38.0 &   \mry{3}{12} &          37.1 \\
                                & Max $\Delta V_\text{tape}$ [V]                                        &         0.265 &         0.256 &         0.249 &         0.243 &               &         0.234 \\
                                & Max $I_\text{tape}$ [A]                                               &          3.86 &          3.75 &          3.67 &          3.60 &               &          3.52 \\ \hline
R0                              & \multirow{6}{*}{Min/Max (LV+HV) Power [W]}                            &   5.11 / 7.27 &   5.12 / 6.85 &   5.14 / 6.52 &   5.16 / 7.00 &   \mry{6}{12} &   5.18 / 5.89 \\
R1                              &                                                                       &   6.12 / 8.01 &   6.14 / 7.65 &   6.17 / 7.38 &   6.19 / 8.23 &               &   6.21 / 6.99 \\
R2                              &                                                                       &   4.01 / 4.83 &   4.02 / 4.68 &   4.03 / 4.70 &   4.04 / 5.40 &               &   4.08 / 4.54 \\
R3                              &                                                                       &   8.96 / 10.7 &   8.99 / 10.4 &   9.02 / 10.8 &   9.05 / 13.1 &               &   9.16 / 10.2 \\
R4                              &                                                                       &   5.04 / 6.06 &   5.06 / 5.88 &   5.08 / 6.12 &   5.09 / 7.17 &               &   5.16 / 5.90 \\
R5                              &                                                                       &   5.57 / 6.63 &   5.59 / 6.45 &   5.61 / 6.42 &   5.63 / 7.16 &               &   5.71 / 6.40 \\ \hline
\multirow{6}{*}{Module-level}   & Max sensor $I_{HV}$ per module [mA]                                   &    0.923 (R3) &     1.77 (R3) &     3.46 (R3) &     7.43 (R3) &   \mry{7}{12} &     1.86 (R3) \\
\multirow{6}{*}{Components}     & Max sensor T [$^\circ$C], Y1                                          &    -27.3 (R3) &    -22.3 (R3) &    -17.3 (R3) &    -12.2 (R3) &               &     7.86 (R3) \\
                                & Max sensor T [$^\circ$C], Y14                                         &    -26.9 (R3) &    -21.5 (R3) &    -15.7 (R3) &    -8.71 (R3) &               &    -26.9 (R3) \\
                                & Max sensor T [$^\circ$C], Max                                         &    -25.8 (R3) &      -21 (R3) &    -15.7 (R3) &    -8.71 (R3) &               &     8.13 (R3) \\
                                & Max $T_\text{Feast}$                                                  &     11.9 (R1) &     14.4 (R1) &     17.5 (R1) &     21.2 (R1) &               &     37.1 (R1) \\
                                & Min $Q_{sensor}$ Headroom [$Q_{S,crit}/Q_{S}$]                        &     7.97 (R3) &     4.49 (R3) &     2.55 (R3) &     1.42 (R3) &               &     4.59 (R3) \\
                                & Min Coolant Temperature Headroom [$^\circ$C]                          &     19.4 (R3) &     14.3 (R3) &     9.07 (R3) &     3.43 (R3) &               &     15.8 (R3) \\ \hline
\multirow{3}{*}{Services}       & Max $\Delta V_\text{HV}$ (filters, \sout{tape}, EOS, cables, PP2) [V] &     5.38 (R1) &     10.3 (R1) &     19.6 (R1) &     38.7 (R1) &   \mry{3}{12} &     10.9 (R1) \\
                                & Max LV round-trip $\Delta V$ from PP2 (type I/II cables only) [V]     &          2.11 &          2.05 &          2.00 &          1.97 &               &          1.93 \\
                                & Max LV $V_\text{out}$ at PP2 [V]                                      &          13.2 &          13.1 &          13.0 &          13.0 &               &          13.0 \\
\hline\end{tabular}
} %% resize box after tabular
\end{centering}
\caption{Summary of nominal (no safety factors) scenarios, with different coolant temperatures.}
\end{subtable}

\vspace{5mm}

\begin{subtable}[t]{.99\linewidth}
\begin{centering}\adjustbox{max width=\textwidth}{ %% just before tabular
\begin{tabular}{|l|l|r|r|r|r|r|r|} \hline % data_below
\multirow{5}{*}{Safety Factors} & Fluence                                                               & \multicolumn{6}{c|}{        0.5} \\
                                & $R_{T}$                                                               & \multicolumn{6}{c|}{        0.2} \\
                                & $I_D$                                                                 & \multicolumn{6}{c|}{        0.2} \\
                                & $I_A$                                                                 & \multicolumn{6}{c|}{       0.05} \\
                                & TID parameterization                                                  & \multicolumn{6}{c|}{pessimistic} \\ \hline
\multirow{2}{*}{HV, Cooling}    & Voltage [V]                                                           &          700.0 &         700.0 &         700.0 &         700.0 &         700.0 &         700.0 \\
                                & Cooling [$^\circ$C]                                                   &     flat $-$35 &    flat $-$30 &    flat $-$25 &    flat $-$20 &    flat $-$15 &    ramp $-$35 \\ \hline
\multirow{3}{*}{Endcap System}  & Total LV+HV, no services                                              &      32.1/39.6 &     32.2/38.5 &               &               &               &     32.9/39.2 \\
\multirow{3}{*}{Min/Max}        &  + type 1 cables, PP1 (Cooling system power)                          &      33.2/41.2 &     33.3/40.0 &   \bf Runaway &   \bf Runaway &   \bf Runaway &     34.1/40.4 \\
\multirow{3}{*}{Power [kW]}     &  + all services and power supplies (Wall power)                       &      45.5/58.3 &     45.7/56.3 &   \bf Year 11 &    \bf Year 7 &    \bf Year 6 &     46.9/54.3 \\
                                & Service power only                                                    &      13.4/18.7 &     13.5/17.8 &      \bf (R3) &      \bf (R3) &      \bf (R3) &     13.4/15.6 \\
                                & Maximum $P_\text{HV}$ [kW]                                            &           3.06 &          5.92 &               &               &               &          6.18 \\ \hline
Petal-level                     & Max petal power (LV+HV) [W]                                           &           53.0 &          52.9 &   \mry{1}{11} &   \mry{1}{ 7} &   \mry{1}{ 6} &          53.7 \\ \hline
\multirow{3}{*}{Petal LV tape}  & Max LV tape power load (incl. tape losses) [W]                        &           50.4 &          48.6 &   \mry{3}{11} &   \mry{3}{ 7} &   \mry{3}{ 6} &          43.7 \\
                                & Max $\Delta V_\text{tape}$ [V]                                        &          0.334 &         0.321 &               &               &               &         0.281 \\
                                & Max $I_\text{tape}$ [A]                                               &           4.72 &          4.57 &               &               &               &          4.13 \\ \hline
R0                              & \multirow{6}{*}{Min/Max (LV+HV) Power [W]}                            &    5.81 / 9.58 &   5.83 / 8.95 &   \mry{6}{11} &   \mry{6}{ 7} &   \mry{6}{ 6} &   5.96 / 7.49 \\
R1                              &                                                                       &    6.99 / 10.6 &   7.03 / 10.1 &               &               &               &   7.21 / 8.90 \\
R2                              &                                                                       &    4.55 / 6.03 &   4.56 / 5.81 &               &               &               &   4.64 / 5.78 \\
R3                              &                                                                       &    10.2 / 13.1 &   10.2 / 14.2 &               &               &               &   10.5 / 14.3 \\
R4                              &                                                                       &    5.78 / 7.50 &   5.80 / 7.41 &               &               &               &   5.92 / 7.75 \\
R5                              &                                                                       &    6.39 / 8.39 &   6.41 / 8.10 &               &               &               &   6.57 / 8.10 \\ \hline
\multirow{6}{*}{Module-level}   & Max sensor $I_{HV}$ per module [mA]                                   &      2.28 (R3) &     5.45 (R3) &   \mry{7}{11} &   \mry{7}{ 7} &   \mry{7}{ 6} &     5.42 (R3) \\
\multirow{6}{*}{Components}     & Max sensor T [$^\circ$C], Y1                                          &     -24.5 (R3) &    -19.4 (R3) &               &               &               &     10.8 (R3) \\
                                & Max sensor T [$^\circ$C], Y14                                         &     -22.7 (R3) &    -15.2 (R3) &               &               &               &    -22.7 (R3) \\
                                & Max sensor T [$^\circ$C], Max                                         &     -21.5 (R3) &    -15.2 (R3) &               &               &               &     11.9 (R3) \\
                                & Max $T_\text{Feast}$                                                  &      49.6 (R1) &     50.8 (R1) &               &               &               &     59.3 (R1) \\
                                & Min $Q_{sensor}$ Headroom [$Q_{S,crit}/Q_{S}$]                        &      2.26 (R3) &     1.25 (R3) &               &               &               &     1.31 (R3) \\
                                & Min Coolant Temperature Headroom [$^\circ$C]                          &      7.44 (R3) &     2.06 (R3) &               &               &               &     2.70 (R3) \\ \hline
\multirow{3}{*}{Services}       & Max $\Delta V_\text{HV}$ (filters, \sout{tape}, EOS, cables, PP2) [V] &      11.8 (R3) &     28.1 (R3) &   \mry{3}{11} &   \mry{3}{ 7} &   \mry{3}{ 6} &     28.0 (R3) \\
                                & Max LV round-trip $\Delta V$ from PP2 (type I/II cables only) [V]     &           2.58 &          2.50 &               &               &               &          2.26 \\
                                & Max LV $V_\text{out}$ at PP2 [V]                                      &           13.7 &          13.6 &               &               &               &          13.3 \\
\hline\end{tabular}
} %% resize box after tabular
\end{centering}
\caption{Summary of worst-case safety factor scenarios, with different coolant temperatures.}
\end{subtable}
\caption{Comparing nominal and worst-case and safety factor scenarios.}
\label{results_summary}
\end{table}
\let\arraystretch\arraystretcha

\restoregeometry
\clearpage

\subsection{Interpretation and Main Takeaways}

Two particularly important quantities to monitor are the $Q_{sensor}$ headroom and the coolant
temperature headroom. The $Q_{sensor}$ headroom is the multiplicative factor
$[Q_{s,critical}/Q_{s,current}]$ by which the sensor $Q$ can increase before reaching thermal runaway
(when this value falls to 1, thermal runaway has occurred). The coolant temperature headroom is the
number of degrees C that you could raise the coolant temperature before thermal runaway occurs (the
headroom falls to 0 at thermal runaway). Even without any safety factors, the $Q_{sensor}$ headroom
is below 10; with all of the safety factors applied, the headroom is around 2. The temperature
headroom tells a similar story.

Two questions that can be asked is how to improve the sensor $Q$ headroom, and, since there are many
moving parts in the model, which methods of improving the sensor $Q$ headroom are most effective? The
answer can be addressed by evaluating the effect of each individual safety factor on the sensor $Q$
headroom (and/or the coolant headroom).

Table~\ref{detailed_safety_table} offers a detailed look at the effect of each individual safety
factor; the ``safety factor scenarios'' are ordered according to which scenario has the larger impact
on the minimum sensor $Q$ headroom (minimum sensor $Q$ headroom means the minimum among all of the sensors in
the endcap, at any point in time). According to the results in the table, the thermal impedance
$R_T$ (with a +20\% safety factor) has the largest impact on the sensor $Q$ headroom,
followed by the HV bias (the safety factor is 700 V), and finally the
currents of the ABC, HCC and AMAC\footnote{
The current is typically dominated by the ABC ($\sim$80\%), followed by the HCC ($\sim$20\%). The AMAC
contributes about 1\% to the total power.}.
Note that the fluence safety factor is out of our control, and the nature of the TID bump does not
have any impact on the sensor $Q$ headroom, since its effect is minimal at the detector end-of-life.

The outlook can be ``improved'' in two ways: first, by improving the accuracy of the safety factors.
In other words, if we can e.g. improve the endcap thermal model and build confidence that the thermal
impedance safety factor should be 10\% rather than 20\%, then our worst-case scenarios will have
larger sensor $Q$ headroom values. Another example would be to guarantee that the HV will never run
above 650V. Of course, the margin for error would not have changed---only our confidence that we can
run within that margin of error.

The outlook can also be improved by physical changes to the petal. This includes pipe rerouting, 
using more thermally conductive material, etc. to reduce the thermal impedance of the material
underneath the sensor. It could include making changes to the ABC chip to reduce the current.
These changes would have the effect of reducing the temperature of the sensors, thus increasing the
sensor $Q$ headroom.

\begin{table}[ht]
\begin{centering}\adjustbox{max width=\textwidth,max totalheight=\textheight}{ %% just before tabular
\begin{tabular}{|ccccc|cc|rrrr|r|r|} \hline % data_below
\multicolumn{5}{|c|}{Safety factor} & $V_{bias}$ & Cooling & Endcaps max & Endcaps max & Min sensor $Q$ & Min Coolant\\
Fluence & $R_{T}$ & $I_D$ & $I_A$ &   TID &   [V] & [$^\circ$C] & \multicolumn{1}{l}{power [kW]} & \multicolumn{1}{l}{HV [kW]} & \multicolumn{1}{l}{headroom} & \multicolumn{1}{l|}{headroom [$^\circ$C]} \\ \hline
0.0       & 0.0       & 0.0       & 0.0        & False      & 500       & $-$35 flat &      32.6 &    1.11 &      7.97 &       19.4 \\
{\bf 0.5} & 0.0       & 0.0       & 0.0        & False      & 500       & $-$35 flat &      33.3 &    1.60 &      5.29 &       15.3 \\
{\bf 0.5} & 0.0       & 0.0       & 0.0        & {\bf True} & 500       & $-$35 flat &      34.9 &    1.60 &      5.29 &       15.3 \\
{\bf 0.5} & 0.0       & {\bf 0.2} & {\bf 0.05} & False      & 500       & $-$35 flat &      37.9 &    1.74 &      4.71 &       14.3 \\
{\bf 0.5} & 0.0       & 0.0       & 0.0        & False      & {\bf 700} & $-$35 flat &      33.4 &    2.37 &      3.76 &       12.0 \\
{\bf 0.5} & {\bf 0.2} & 0.0       & 0.0        & False      & 500       & $-$35 flat &      32.9 &    1.84 &      3.66 &       11.9 \\
{\bf 0.5} & {\bf 0.2} & {\bf 0.2} & {\bf 0.05} & {\bf True} & {\bf 700} & $-$35 flat &      39.6 &    3.06 &      2.26 &       7.44 \\ \hline
0.0       & 0.0       & 0.0       & 0.0        & False      & 500       & $-$30 flat &      31.8 &    1.95 &      4.49 &       14.3 \\
{\bf 0.5} & 0.0       & 0.0       & 0.0        & False      & 500       & $-$30 flat &      32.3 &    2.93 &      2.95 &       10.2 \\
{\bf 0.5} & 0.0       & 0.0       & 0.0        & {\bf True} & 500       & $-$30 flat &      33.8 &    2.93 &      2.95 &       10.2 \\
{\bf 0.5} & 0.0       & {\bf 0.2} & {\bf 0.05} & False      & 500       & $-$30 flat &      36.7 &    3.21 &      2.63 &       9.10 \\
{\bf 0.5} & 0.0       & 0.0       & 0.0        & False      & {\bf 700} & $-$30 flat &      32.5 &    4.33 &      2.10 &       6.87 \\
{\bf 0.5} & {\bf 0.2} & 0.0       & 0.0        & False      & 500       & $-$30 flat &      32.1 &    3.45 &      2.03 &       6.65 \\
{\bf 0.5} & {\bf 0.2} & {\bf 0.2} & {\bf 0.05} & {\bf True} & {\bf 700} & $-$30 flat &      38.5 &    5.92 &      1.25 &       2.06 \\ \hline
0.0       & 0.0       & 0.0       & 0.0        & False      & 500       & $-$25 flat &      31.7 &    3.56 &      2.55 &       9.07 \\
{\bf 0.5} & 0.0       & 0.0       & 0.0        & False      & 500       & $-$25 flat &      33.7 &    5.61 &      1.63 &       4.67 \\
{\bf 0.5} & 0.0       & 0.0       & 0.0        & {\bf True} & 500       & $-$25 flat &      33.7 &    5.61 &      1.63 &       4.67 \\
{\bf 0.5} & 0.0       & {\bf 0.2} & {\bf 0.05} & False      & 500       & $-$25 flat &      38.1 &    6.23 &      1.44 &       3.50 \\
{\bf 0.5} & 0.0       & 0.0       & 0.0        & False      & {\bf 700} & $-$25 flat &      36.7 &    8.56 &      1.16 &       1.39 \\
{\bf 0.5} & {\bf 0.2} & 0.0       & 0.0        & False      & 500       & $-$25 flat &      35.3 &    7.02 &      1.09 &      0.854 \\
{\bf 0.5} & {\bf 0.2} & {\bf 0.2} & {\bf 0.05} & {\bf True} & {\bf 700} & $-$25 flat & \multicolumn{4}{c|}{\bf Runaway Year 11} \\
\hline\end{tabular}
} %% resize box after tabular
\caption{Summary of all safety factor scenarios and their effect on the maximum power in the
endcaps (excluding services), the maximum HV power, the minimum sensor headroom $[Q_{s,crit}/Q_s]$, and the coolant temperature headroom.}
\label{detailed_safety_table}
\end{centering}
\end{table}
